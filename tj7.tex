\documentclass[UTF8,a4paper,11pt]{ctexart}
\usepackage[margin=1in,xetex]{geometry}
\usepackage{listings} 
\usepackage{xcolor} 
\usepackage{amsmath}
\usepackage{amssymb}
\newtheorem{definition}{定义}
\newtheorem{theorem}{定理}
\newtheorem{proof}{证明}
\newtheorem{lemma}{引理}
\lstset{
  basicstyle=\tt,
  keywordstyle=\color{purple}\bfseries,
  identifierstyle=\color{brown!80!black},
  commentstyle=\color{gray},
  showstringspaces=false,
  numbers=left,                
  numberstyle=\small,               
}
\title{微分方程笔记}
\author{5eqn}
\date{\today}
\begin{document}
  \maketitle
  \section{可分离变量的微分方程}
    一阶微分方程通常可以分离变量变成
    \[
    \begin{aligned}
    g\left(y\right)\mathrm{d}y=f\left(x\right)\mathrm{d}x.
    \end{aligned}
    \]
    
    例如对
    \[
    \begin{aligned}
      \frac{\mathrm{d}y}{\mathrm{d}x}=2xy,
    \end{aligned}
    \]
    可以变形为
    \[
    \begin{aligned}
      \int \frac{\mathrm{d}y}{y}=2x \mathrm{d}x,
    \end{aligned}
    \]
    解得
    \[
    \begin{aligned}
      \ln \left| y \right|=x^{2}+C_1.
    \end{aligned}
    \]
  \section{齐次方程}
    对于齐次方程
    \[
    \begin{aligned}
      \frac{\mathrm{d}y}{\mathrm{d}x}=\varphi\left(\frac{y}{x}\right),
    \end{aligned}
    \]
    存在一种巧妙的换元方法把$\frac{\mathrm{d}y}{\mathrm{d}x}$拆开, 即
    \[
    \begin{aligned}
      y=ux.
    \end{aligned}
    \]
    
    换元后, 方程变形为
    \[
    \begin{aligned}
      u+x \frac{\mathrm{d}u}{\mathrm{d}x}=\varphi\left(u\right),
    \end{aligned}
    \]
    便可以按照可分离变量的微分方程求解.

    同时, 可以通过对$x$和$y$线性偏移, 例如令$u=x+1$, 来实现齐次化.

    例如对
    \[
    \begin{aligned}
      \left(2x+y-4\right)\mathrm{d}x+\left(x+y-1\right)\mathrm{d}y=0,
    \end{aligned}
    \]
    令
    \[
    \begin{aligned}
      \begin{cases}
        x=X+h,\\y=Y+k,
      \end{cases}
    \end{aligned}
    \]
    得
    \[
    \begin{aligned}
      \begin{cases}
        h=3,\\k=-2,
    因此令
      \end{cases}
    \end{aligned}
    \]
    因此原方程变为
    \[
    \begin{aligned}
      \frac{\mathrm{d}Y}{\mathrm{d}X}&=-\frac{2+\frac{Y}{X}}{1+\frac{Y}{X}},\\
      X \frac{\mathrm{d}u}{\mathrm{d}X}&=-\frac{2+u}{1+u}-u,\\
      -\frac{u+1}{u^{2}+2u+2}\mathrm{d}u&=\frac{\mathrm{d}X}{X},\\
      2x^{2}+2xy+y^{2}-8x-2y&=C
    \end{aligned}
    \]
  \section{一阶线性微分方程}
    \subsection{原本线性}
      对于
      \[
      \begin{aligned}
        \frac{\mathrm{d}y}{\mathrm{d}x}+P\left(x\right)y=Q\left(x\right),
      \end{aligned}
      \]
      $Q\left(x\right)=0$的解$y=Ce^{-\int P\left(x\right)\mathrm{d}x}$
      告诉我们若将$C$视为函数, 则右侧部分能消掉$P\left(x\right)y$项.
      为了在通用的情况下也消除掉这一项, 不妨作换元
      \[
      \begin{aligned}
        y=ue^{-\int P\left(x\right)\mathrm{d}x},
      \end{aligned}
      \]
      那么首先$Q\left(x\right)=0$时$u=C$,
      并且
      \[
      \begin{aligned}
        \frac{\mathrm{d}y}{\mathrm{d}x}&=u^{\prime}e^{-\int P\left(x\right)\mathrm{d}x}-uP\left(x\right)e^{-\int P\left(x\right)\mathrm{d}x},
      \end{aligned}
      \]
      代入原微分方程, 会发现$uP\left(x\right)e^{-\int P\left(x\right)\mathrm{d}x}$
      真的被消掉了, 因此
      \[
      \begin{aligned}
        u^{\prime}e^{-\int P\left(x\right)\mathrm{d}x}&=Q\left(x\right),
      \end{aligned}
      \]
      最终
      \[
      \begin{aligned}
        u&=\int Q\left(x\right)e^{\int P\left(x\right)\mathrm{d}x}\mathrm{d}x+C,\\
        y&=Ce^{-\int P\left(x\right)\mathrm{d}x}+e^{-\int P\left(x\right)\mathrm{d}x}\int Q\left(x\right)e^{\int P\left(x\right)\mathrm{d}x}\mathrm{d}x.
      \end{aligned}
      \]

      为了方便记忆, 可以考虑着重记忆$y$到$u$的换元方式 (这是为了消掉$P\left(x\right)y$项),
      以及换元后乘积求导只有$u^{\prime}$项被保留的事实 (另一半用来消掉$P\left(x\right)y$项),
      就能较快推出最后的结果.

      例如对
      \[
      \begin{aligned}
        \frac{\mathrm{d}y}{\mathrm{d}x}-\frac{2y}{x+1}=\left(x+1\right)^{\frac{5}{2}},
      \end{aligned}
      \]
      那么关键步骤为令
      \[
      \begin{aligned}
        y=u\left(x+1\right)^{2},
      \end{aligned}
      \]
      以及代入消除后
      \[
      \begin{aligned}
        u^{\prime}=\sqrt{x+1},
      \end{aligned}
      \]
      因此
      \[
      \begin{aligned}
        y=\left(x+1\right)^{2}\left(\frac{2}{3}\left(x+1\right)^{\frac{3}{2}}+C\right).
      \end{aligned}
      \]
    \subsection{伯努利方程}
      方程
      \[
      \begin{aligned}
        \frac{\mathrm{d}y}{\mathrm{d}x}+P\left(x\right)y=Q\left(x\right)y^{n} && \left(n\neq 0, 1\right)
      \end{aligned}
      \]
      是伯努利方程,
      若要转化成线性方程,
      考虑将右侧形式统一化, 变为
      \[
      \begin{aligned}
        y^{-n}\frac{\mathrm{d}y}{\mathrm{d}x}+P\left(x\right)y^{1-n}=Q\left(x\right),
      \end{aligned}
      \]
      再令$k=1-n$即可.

  \section{可降阶的高阶微分方程}
    \subsection{单个高阶导}
      对于
      \[
      \begin{aligned}
        y^{(n)}=f\left(x\right),
      \end{aligned}
      \]
      只要对右侧不断积分即可.

    \subsection{两项阶数相邻}
      对于
      \[
      \begin{aligned}
        y^{\prime\prime}=f\left(x, y^{\prime}\right),
      \end{aligned}
      \]
      按照正常的思路解出$y^{\prime}$后再算$y$即可.

  \section{高阶线性微分方程}
    对于线性微分方程, 可以类比线性代数, 
    得到齐次方程通解具备线性, 非齐次方程的特解和齐次方程解空间基底均正交,
    即设齐次方程解空间基底为$y_1, y_2$, 非齐次方程特解为$y^{*}$,
    那么
    \[
    \begin{aligned}
      y=C_1y_1+C_2y_2+y^{*}.
    \end{aligned}
    \]
    
    在求出齐次方程解空间基底后, 依然可以采用常数变易法,
    例如对
    \[
    \begin{aligned}
      y^{\prime\prime}-2y^{\prime}+y=0,
    \end{aligned}
    \]
    已知齐次方程解空间基底为$y_1=e^{x}$,
    那么令$y=e^{x}u, y^{\prime}=e^{x}\left(u^{\prime}+u\right), y^{\prime\prime}=e^{x}\left(u^{\prime\prime}+2u^{\prime}+u\right)$,
    由于保证消掉$y$的低阶导, 我们得到
    \[
    \begin{aligned}
      e^{x}u^{\prime\prime}=\frac{1}{x}e^{x},
    \end{aligned}
    \]
    因此$u=C_1+C_2x+x \ln \left| x \right|$.

  \section{常系数齐次线性微分方程}
    对于
    \[
    \begin{aligned}
      y^{\prime\prime}+py^{\prime}+qy=0,
    \end{aligned}
    \]
    不妨令$y=e^{rx}$,
    那么
    \[
    \begin{aligned}
      \left(r^{2}+p r+q\right)e^{rx}=0,
    \end{aligned}
    \]
    即
    \[
    \begin{aligned}
      r^{2}+p r+q=0.
    \end{aligned}
    \]
    
    对于能解出两个根的情况, 直接代入即可,
    其中如果是复数要运用公式
    \[
    \begin{aligned}
      e^{\left(\alpha+\beta i\right)x}=e^{\alpha x}\left(\cos \beta x+i \sin \beta x\right)
    \end{aligned}
    \]
    变形得到最终答案.

    对于重根, 使用常数变易法设$y=e^{rx}u\left(x\right)$,
    那么
    \[
    \begin{aligned}
      \left(u^{\prime\prime}+2ru^{\prime}+r^{2}u\right)+p\left(u^{\prime}+ru\right)+qu=0,
    \end{aligned}
    \]
    由于重根, 化简得$u^{\prime\prime}=0$, 而$u$不为常数,
    则令$u=x$即可. 注意这里采用$u=x+1$也可以, 而且可能更像 (但不是) 指数函数退化.

    例如对于有阻尼振动, 我们有
    \[
    \begin{aligned}
      x^{\prime\prime}+2nx^{\prime}+k^{2}x=0
    \end{aligned}
    \]
    特征方程根为
    \[
    \begin{aligned}
      -n \pm \sqrt{n^{2}-k^{2}},
    \end{aligned}
    \]
    那么在小阻尼$n<k$时, 令$\omega=\sqrt{k^{2}-n^{2}}$, 那么
    \[
    \begin{aligned}
      x=e^{-nt}\left(x_0 \cos \omega t + \frac{v_0+nx_0}{\omega}\sin \omega t\right),
    \end{aligned}
    \]
    令
    \[
    \begin{aligned}
      x_0=A \sin \varphi, &&
      \frac{v_0+nx_0}{\omega}=A \cos \varphi \left(0\le \varphi \le 2\pi\right),
    \end{aligned}
    \]
    那么
    \[
    \begin{aligned}
      x=Ae^{-nt}\sin\left(\omega t+\varphi\right),
    \end{aligned}
    \]
    其中
    \[
    \begin{aligned}
      A=\sqrt{x_0^{2}+\frac{\left(x_0+nx_0\right)^{2}}{\omega^{2}}},&&
      \tan \varphi = \frac{x_0\omega}{v_0+nx_0}.
    \end{aligned}
    \]

    在临界阻尼$n=k$时, 特征方程重根, 因此结果为
    \[
    \begin{aligned}
      x=e^{-nt}\left(x_0+\left(v_0+nx_0\right)t\right).
    \end{aligned}
    \]
    
    在大阻尼$n>k$时, 特征方程有两负根, 即
    \[
    \begin{aligned}
      x=C_1e^{-\left(n-\sqrt{n^{2}-k^{2}}\right)t}+C_2e^{-\left(n+\sqrt{n^{2}-k^{2}}\right)t}.
    \end{aligned}
    \]
    
    对于更多元的情况, 可能出现更多重根, 这时堆叠$x$的幂即可.
    
  \section{常系数非齐次线性微分方程}
    正如前面所提到, 采用常数变易法 (但是教材上变成了待定系数法) 即可.
  \section{欧拉方程}
    对于
    \[
    \begin{aligned}
      x^{n}y^{(n)}+p_1x^{n-1}y^{(n-1)}+\cdots +p_{n-1}xy^{\prime}+p_ny=f\left(x\right),
    \end{aligned}
    \]
    想办法令$\frac{\mathrm{d}y}{\mathrm{d}x}=\frac{1}{x}\frac{\mathrm{d}y}{\mathrm{d}t}$,
    即$t=\ln x$即可.

    注意到在此之后
    \[
    \begin{aligned}
      x^{k}y^{(k)}=D\left(D-1\right)\cdots \left(D-k+1\right)y.
    \end{aligned}
    \]
  \section{常系数线性微分方程组}
    通过消元变成前面学过的即可.
\end{document}
